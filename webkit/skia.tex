\documentclass[a4paper,dvipdfm]{article}
%------------------------------------------------------------------------------------------------------
%\usepackage[symbol]{footmisc}这个宏包的作用是更改脚注的一些排版方式选项有: perpage stable side multiple para symbol ragged marginal flushmargin hang
%\usepackage{abstract}%这个红包的作用是双栏排版时,让摘要不这样,使用onecolabstract环境即可
\usepackage{booktabs}%for \toprule \midrule \bottomrule
\usepackage{fix2col}%修补在双栏排版时无法保证浮动体的排印顺序
\usepackage{fontspec,xltxtra,xunicode}    % 这里使用 xetex 相应的宏包
\usepackage{indentfirst} 					%缩进章节或者section里的第一段
\usepackage{url}
\usepackage{natbib} %cite URL
\usepackage{graphicx}
\usepackage{epsfig}
\usepackage[left=1.8cm,right=1.8cm,top=2cm,bottom=1.5cm]{geometry}
%\usepackage[left=0.1in, right=0.1in,  top=0.1in, bottom=0.1in, papersize={91.44mm, 121.92mm}]{geometry} %this is for kindle ebook
\usepackage{zhfont}                       % 这里调用 zhfont.sty 
%\usepackage[left=1.8cm,right=1.8cm,top=2cm,bottom=1.5cm]{geometry}
%\setzhmainfont{Microsoft JhengHei}
\setmainfont{Georgia}
\zhspacing

%以下是把论文中的一些参数中文化,比如摘要、参考文献等
\XeTeXlinebreaklocale “zh”
\XeTeXlinebreakskip = 0pt plus 1pt
\renewcommand\arraystretch{1.5}
\renewcommand{\contentsname}{目录}
\renewcommand{\listfigurename}{插图目录}
\renewcommand{\listtablename}{表格目录}
\renewcommand{\refname}{参考文献}
\renewcommand{\abstractname}{摘要}


\usepackage{tipa}
\usepackage{textcomp}





\pagestyle{empty}

\begin{document}
\title{Skia}
\author{webkit task force}
\maketitle

\section{硬件加速}
Skia软件绘图和使用opengl的硬件绘图是使用SkCanvas类作为绘图接口,区别在于软件绘图是将图形绘制到SkBitmap上,而硬件绘图则是直接将图形绘制到屏幕上。
Skia的软件绘图和硬件加速的初始化方式

\begin{enumerate}
	\item 软件绘图:SkCanvas canvas(new SkDevice(bitmap));
	\item 硬件绘图:SkCanvas* canvas = new SkCanvas(GrContext, GrRenderTarget);

			GrContext是和gpu交互的公共接口,管理着gpu上下文,纹理及字体缓存.

			GrRenderTarget是绘制的目标,其中保存了需要绘制的大小,帧缓冲区等内容
\end{enumerate}

Skia的canvas绘制是根据SkCanvas中的保存的device对象来确定是软件绘制还是硬件绘制。

\section{SkCanvas}
SkCanvas是Skia的一个核心类,他封装了所有对设备进行的画图操作。这个类自身包含了一个设备的引用,以及一个矩阵和裁剪栈。所有的画图操作,都是在 经过栈内存放的矩阵变幻之后才进行的。当然,最终显示给用户的图像,还必须经过裁剪堆栈的运算。SkCanvas记录着整个设备的绘画状态,而设备上面绘制的对象的状态又是由SkPaint类来记录的,SkPaint类作为参数,传递给不同 SkCanvas类的成员函数。SkPaint类里记录着如颜色, 字体, 文字大小, 文字粗细, 渐变等信息。

\begin{enumerate}
	\item 构造函数,给定一个Bitmap或者Device,在给定的这个对象上进行画图

		explicit SkCanvas(SkDevice* device);

		explicit SkCanvas(const SkBitmap\&  bitmap);

	\item 移位,缩放,旋转,变形函数

		virtual bool translate(SkScalar dx, SkScalar dy);

		virtual bool scale(SkScalar sx, SkScalar sy);

		virtual bool rotate(SkScalar degrees);

		virtual bool skew(SkScalar sx, SkScalar sy);
	\item 画布内画图函数

		void drawARGB(U8CPU a, U8CPU r, U8CPU g, U8CPU b,SkXfermode::Mode mode = SkXfermode::kSrcOver\_Mode);

		给定透明度以及RGB3色,填充整个可绘制区域。

		void drawColor(SkColor color, SkXfermode::Mode mode = SkXfermode::kSrcOver\_Mode);

		给定颜色color, 填充整个绘制区域。

		virtual void drawPaint(const SkPaint\& paint);

		用指定的画笔填充整个区域。

		virtual void drawPoints(PointMode mode, size\_t count, const SkPoint pts[],const SkPaint\& paint);

	void drawPoint(SkScalar x, SkScalar y, const SkPaint\& paint);

	void drawPoint(SkScalar x, SkScalar y, SkColor color);

	根据各种不同参数绘制不同的点。

	void drawLine(SkScalar x0, SkScalar y0, SkScalar x1, SkScalar y1, const SkPaint\& paint);

	画线,起点(x0, y0), 终点(x1, y1), 使用paint作为画笔。

	virtual void drawRect(const SkRect\& rect, const SkPaint\& paint);

	画矩形,矩形大小由rect指定,画笔由paint指定。

	void drawRectCoords(SkScalar left, SkScalar top, SkScalar right, SkScalar bottom, const SkPaint\& paint);

	给定4个边界画矩阵。

	void drawOval(const SkRect\& oval, const SkPaint\&);

	画椭圆,椭圆大小由oval矩形指定。

	void drawCircle(SkScalar cx, SkScalar cy, SkScalar radius, const SkPaint\& paint);

	给定圆心坐标和半径画圆。

	void drawArc(const SkRect\& oval, SkScalar startAngle, SkScalar sweepAngle,bool useCenter, const SkPaint\& paint);

	画弧线,用法类似于画椭圆。

	void drawRoundRect(const SkRect\& rect, SkScalar rx, SkScalar ry, const SkPaint\& paint);

	画圆角矩形,x, y方向的弧度用rx, ry指定。

	virtual void drawPath(const SkPath\& path, const SkPaint\& paint);

	路径绘制,根据path指定的路径绘制路径。

	virtual void drawBitmap(const SkBitmap\& bitmap, SkScalar left, SkScalar top, const SkPaint* paint = NULL);

	绘制指定的位图。

	virtual void drawText(const void* text, size\_t byteLength, SkScalar x, SkScalar y, const SkPaint\& paint);

	以(x,y)为起始点写文字,文字存储在text指针内,长度有byteLength指定。

	Skia对于字体的绘制是根据不同平台实现的,对于不同平台使用不同的字体引擎,并从字体引擎获得字体图片进行渲染。

\end{enumerate}
\end{document}

